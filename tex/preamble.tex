% !TEX root = ../thesis.tex
\begin{preamble}

\iffinal{}{\newpage}

\begin{DUTdedications}
\begin{center}
To Nicole.
\end{center}
\end{DUTdedications}

\iffinal{}{\newpage}

\begin{acknowledgments}
First and foremost, I'd like to thank and acknowledge my thesis advisor Joshua Jacobs. I've known Josh since I was an undergraduate and he was a graduate student at the University of Pennsylvania, over a decade ago. Through all that time, even well before I knew I'd be his student, Josh has been a source of knowledge and guidance. Josh has taught me how to be a successful scientist, and, moreover, Josh has given me a set of skills that I am confident will serve me well in the future. I'm grateful to call Josh my friend, and I'm looking forward to our continued friendship over the years to come.

I am also indebted to Michael Kahana, who I've worked with for many years, and with whom I still collaborate. It was because of Mike that I first became interested in neuroscience, and it was through working in his lab as an undergraduate that I first began to understand the excitement that comes from trying to unravel the mysteries of the human brain. I would also like to thank many current and former members of the Kahana lab, in particular those who have helped me with the arduous task of data collection, including Emily Rosenberg, Erin Beck, Ryan Baily Williams, Patrick Crutchley, Ashwin Ramayya, John Burke, Deb Levy, and Anastasia Lyalenko. In addition, I'm grateful for my current lab members, notably Tom Coffey and Sang Ah Lee, for their invaluable support and insights over the course of graduate career.

These acknowledgments would be incomplete without mentioning my clinical collaborators at Thomas Jefferson University Hospital, the Hospital of the University of Pennsylvania, UCLA Medical Center, and Freiburg University Hospital. It is through their efforts that I am afforded the rare opportunity to collect human neural recordings. I'd also like to thank the patients themselves from whom I've collected data, who selflessly volunteer to participate in our studies and who make this entire endeavor possible.

Lastly, I'd like to thank my wife Nicole Long, whose love and support has kept me focused, and my parents, for their unending encouragement.


\end{acknowledgments}

\iffinal{}{\newpage}

\tableofcontents 
\iffinal{}{\newpage}

\listoftables
\iffinal{}{\newpage}

\listoffigures 
\iffinal{}{\newpage}

\begin{abstract}
\ifdaring{\setstretch{1.3}}{\setstretch{1.6}}

The ability to navigate our environment is a vital skill for numerous species, including humans. How does the brain encode external space to allow for accurate navigation? Moreover, as we move through the world, how do we keep track of where specific events occur? Based on decades of research in rodents, we know that that hippocampus contains \textit{place cells} that code for particular locations in the environment, and based on decades of work in humans, we know that the hippocampus is crucial for episodic memory function. The goal of this dissertation is to study how the human brain simultaneously supports spatial navigation and memory function by analyzing intracranially recorded EEG from participants performing virtual spatial memory tasks. In my first study, I investigated whether the neural representation of space formed by the place cell population code in the medial temporal lobe (MTL) becomes integrated with a broader memory signal. I found that place cells in human MTL act as mechanism for memories to become linked to the location where they occurred, suggesting that the neural system underlying spatial navigation and neural system underlying memory function are not as distinct as once thought.  In my next study, I investigated whether anatomical subregions of the human MTL, specifically the entorhinal cortex (EC) and the hippocampus, differ in the type of spatial information they are selective to, which has been shown to be true in rodents. I discovered a new type of cell in the human EC called \textit{path equivalent cells} that provide a metric of distance relative to an environment's geometry, unlike  hippocampal place cells that only fire at specific locations. This findings helps to bring our understanding of how space is represented in the human brain closer to our understanding of space in the rodent brain, which has been studied for many more decades. In my final study, I investigated how oscillatory activity in the human hippocampus was modulated by movement through the environment. In rodents, the \textit{theta} oscillation (4--8 Hz) is closely linked to voluntary movement through space and is an integral component for many rodent derived theories of MTL function. I found that functionally analogous signals in human hippocampus appeared at lower frequencies than in rodents, suggesting that these theories may require modification before they can be broadly applied to other species. Taken together, my work helps to reconcile how the MTL supports both spatial navigation and episodic memory function, as well as bridging the gap between the large literature describing the neural representation of space in the rodent brain and the comparatively less well understood mechanisms in the human brain.
\end{abstract}

\iffinal{}{\newpage}

\end{preamble}


























