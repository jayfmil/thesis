% !TEX root = ../thesis.tex
\chapter{General discussion}
\large

\section{Summary}

In this dissertation, I recorded intracranial brain activity as patients traveled through carefully designed 3D virtual environments in order to investigate the neural signals related to spatial navigation and memory function. My work has shown that the neural mechanisms responsible for representing external space are also involved in the storage and retrieval of episodic memories, helping to reconcile how the hippocampus and medial temporal lobe (MTL) can simultaneously be vital for both navigation and declarative memory abilities. I have also advanced our understanding of exactly how space is coded in the human brain using both single unit and oscillatory analyses, and I provide insights into how this neural system aligns with or differs from the more well characterized rodent brain. 

In Chapter 2, I isolated place cells in the human MTL and examined whether their functional role extends beyond simply supporting a representation of space. Place cells have been identified in humans in only a small number of previous studies \citep{EkstEtal03,JacoEtal10}, thus the identification of place cells was a crucial first step in my analysis. We found that 



% In the rodent literature, there as been a longstanding debate about the exact 


% josh mentioned theta, inter region communication
