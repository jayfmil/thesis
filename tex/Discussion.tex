% !TEX root = ../thesis.tex
\chapter{General discussion}
% \large

\section{Summary}

In this dissertation, I recorded intracranial brain activity as patients traveled through carefully designed 3D virtual environments in order to investigate the neural signals related to spatial navigation and memory function. My work shows that the neural mechanisms responsible for representing external space are also involved in the storage and retrieval of episodic memories, helping to reconcile how the hippocampus and medial temporal lobe (MTL) can simultaneously be vital for both navigation and declarative memory abilities. I have also advanced our understanding of exactly how space is coded in the human brain using both single unit and oscillatory analyses, and I provide insights into how this neural system aligns with or differs from that in the more well characterized rodent brain. 

In Chapter 2, I isolated place cells in the human MTL and examined whether their functional role extends beyond simply supporting a representation of space. Place cells have been identified in humans in only a small number of previous studies \citep{EkstEtal03,JacoEtal10}, thus the identification of place cells was a crucial first step in my analysis. We found that $\sim$25\% of MTL neurons had well defined place fields, such that they exhibited significantly elevated rates of spiking within the field compared to other areas of the virtual environment. Based upon models of human episodic memory function, we hypothesized that these place cells would not only be involved in coding for space, but should also reactivate when patients recalled events that occurred within the virtual environment. This hypothesis is grounded in a theoretical account of human memory known as \textit{retrieved context}, which argues that when a memory is stored, it is stored along with its associated context. Here, context refers to an amalgam of both internal and external information present when an event occurs \citep{McGe42,Bowe72}, including, but not limited to, space and time (hence, the term spatiotemporal context). When a memory is later recalled, the theory posits that the context is recovered as well \citep{HowaKaha02a,PolyEtalTulv,LohnKaha13a}. If place cells represent not just an internal map, but, moreover, the \textit{spatial context} of a memory, then the pattern of place cell activity that represents a particular location should reemerge when events that occurred at that location are later recalled, even when the participant is no longer navigating the environment. Indeed, this is precisely what we find, such that place cell activity prior to and during the vocalization of a memory closely resembles place cell activity that codes for the location where the memory was formed.

The work described in Chapter 2 capitalized on the knowledge that place cells exist in human MTL and exhibit similar behavior to place cells found in rodents. However, our understanding of the intricacies of the neural representation of space in the human brain still lags far behind that of rodents, where numerous types of spatially tuned cells have been found and the distinctive functions of MTL subregions have been fairly well characterized (for a review of the rodent literature, see \cite{MoseEtal08}). In Chapter 3, I attempted to partially close this gap by identifying additional classes of specialized cells beyond place cells in humans, as well as by highlighting key differences in the type of information coded by distinct MTL subregions. We first isolated cells in the MTL whose firing rates varied as a function of location within the virtual environment, and then we further sorted these cells based on the precise characteristics of their firing patterns. We found that cells predominantly located in the entorhinal cortex (EC) coded for multiple related locations in the virtual environment, in contrast to the standard place cell that  fires at a single location. More specifically, these EC cells exhibited multiple firing fields such that individual cells spiked at the same relative location across multiple distinct regions of the environment, seemingly coding a measure of relative distance that is tied to environmental geometry. We labeled these cells \textit{path equivalent} cells, as their properties are very similar to path equivalent cells previously found in rodent EC \citep{FranEtal00}. This finding nicely parallels the rodent literature, where there exists a diversity of cell classes in the EC that code for numerous environmental features \citep{HaftEtal05,FyhnEtal04,DerdEtal09,SargEtal06,SolsEtal08}.

In Chapter 4, I continued my comparison between rodent and human data, moving from analyses of neuronal spiking to analyses of oscillatory activity. Focusing solely on the hippocampus, I investigated whether the \textit{theta} oscillation, which plays so prominent a role in rodent spatial navigation, was similarly modulated by movement during human navigation. Though previous human studies have shown a general increase in low frequency hippocampal activity during periods of virtual movement \citep{CaplEtal03,EkstEtal05,JacoEtal10c}, my task was explicitly designed to test for the effects movement speed on neural activity, which is tightly coupled to theta power in the rodent \citep{Vand69}. I found that movement related oscillatory activity in the human hippocampus is consistently expressed at frequencies between 2 and 4 Hz, unlike the functionally analogous rodent signal which appears at $\sim$7--8 Hz \citep{Buzs02}. Moreover, I showed that the amplitude of the theta signal increases as a function of movement speed, replicating the longstanding rodent phenomenon \citep{McFaEtal75}. Beyond this amplitude/speed relationship, the fact that I did not find a corresponding increase in the frequency of theta associated with faster movements could perhaps be a result of the virtual environment, where the frequency/speed relationship normally present in rodents does not transfer from real world to virtual testing \citep{RavaEtal13}. In addition to movement related oscillatory activity, I also examined the relationship between low frequency activity and memory for object locations within the virtual environment. I found that a ``slow theta'' signal present at $\sim$3 Hz was correlated with better memory performance whereas a ``fast theta'' at $\sim$7 Hz was correlated with worse memory performance, building on previous work showing that grouping signals into their traditional frequency bands may mask underlying effects \citep{LegaEtal12}.

% type 1 vs type 2 theta?

% These results are consistent with converging findings indicating that human hippocampal theta is present at lower frequencies than in rodents \citep{WatrEtal13a,Jaco14}, and raises concerns about the applicability of rodent-derived models of brain function that rely the precise frequency of the theta oscillation \citep{JensLism98,BurgEtal07}.


\section{Regional differences in medial temporal lobe spatial coding}

What are the distinctive roles of various MTL subregions, in particular, the hippocampus and the entorhinal cortex, in coding for space? As mentioned in Chapter 2 and highlighted in Chapter 3, there are many different types of spatially tuned cells. Hippocampal place cells create a seemingly allocentric (map based) representation of external space \citep{OKeeDost71,Mull96}. This representation is formed based upon features of the environment, such as salient landmarks \citep{OKeeBurg96}. Notably, when a rodent is moved between environments or aspects of the current environment are changed, the hippocampal spatial code undergoes a process known as \textit{remapping} \citep{MullKubi87}. When remapping occurs, a place cell that was active in one environment may not be active in another, or it may alter the location of its place field to represent a completely new area of space \citep{MarkEtal95a,LeutEtal04a,LeutEtal05}. In contrast, cells in upstream brain regions, like the EC, tend to provide a quite different sort of information, such that they do not code for precise locations in specific environments. Rather, the activity of these cells is largely invariant to the specific environmental features. Grid cells that are co-active in one environment are likely to be co-active in another and will maintain the spacing of their firing locations \citep{FyhnEtal07}. Similarly, the population activity of head direction cells is coherent across environments \citep{TaubEtal90}, and, likewise, a border cell that fires along the west wall of one environment will continue to fire along the west wall of a new environment \citep{SolsEtal08}. Thus, these cells are believed to provide part of the neural metric for calculating location based on self-motion, or, in other words, the ability to perform path integration independent of particular environmental features \citep{JeffBurg06,BuzsMose13}.

My work discovering path equivalent cells in the human EC is one of the few attempts to quantify the functional differentiation between human MTL regions at the level of single neurons, though the literature is beginning to grow. In addition to place cells in hippocampus \citep{EkstEtal03},``path cells'' have been found in human EC that code for circular direction of travel \citep{JacoEtal10}, and human analogues of EC grid cells have recently been located as well \citep{JacoEtal13}. Now that a diverse spectrum of cells classes have been uncovered in both rodents and, to a lesser but increasing degree, in humans, a recent topic of great interest is delving into what aspects of the hippocampal place code are internally generated and what aspects rely on EC input. Given that many EC neurons synapse directly onto hippocampal place cells in rodents \citep{ZhanEtal13}, it is believed that the location specificity of place cells in the hippocampus is facilitated in part by the transmission of a broad array of EC input \citep{MoseMose13}. However, it has also been shown that grid cells, which are the most common type of spatially sensitive cell type found in EC, develop later in the lifecyle of rat pups than do place cells \citep{LangEtal10,WillEtal10}, and, moreover, place cell activity can persist even in the presence of EC lesions, though it is diminished \citep{HaleEtal14}. The exact role that each type of cell class, such as the path equivalent cells I've reported, may play in the formation of the place code is an open area of investigation \citep{BushEtal14}.

% This hypothesis is compatible with my findings, though the exact mechanisms of such a transformation are not yet understood.

% Moreover, if EC input aids in the generation of place cells, how are the non-specific EC signals transformed into the precise firing of hippocampal cells?

% The method through which non-specific EC signals are transformed into the precise firing of hippocampal cells is a current topic of great interest.


\section{Beyond space in the hippocampus}
In the rodent literature, there has been a longstanding debate about the exact nature of the hippocampus. Namely, does the hippocampus act, fundamentally, as cognitive map that encodes the surrounding spatial features, or has the fact that many hippocampal cells are so strongly tuned to space caused researchers to overlook non-spatial coding properties of the hippocampus \citep{EichEtal99}? The cognitive mapping hypothesis, popularized by \citet{OKeeNade78} held great sway in the field for a number of decades, but more recent work has begun to reveal that cells in the rodent hippocampus are modulated by many variables beyond spatial information. Hippocampal cells have been shown to respond to non-spatial factors such as texture, odor, and color \citep{YounEtal94,WoodEtal99,LeutEtal05,IgarEtal14}. Moreover, the firing patterns of place cells are are affected by task demands, such that they distinguish between future goal locations \citep{WoodEtal00,FerbShap03}, conjunctively code for item and place information \citep{KomoEtal09}, and can keep track of temporal order \citep{MannEtal07}. Intriguingly, newly discovered hippocampal \textit{time cells} provide a mechanism for self-localizing in time in addition to self-localizing in space \citep{MacDEtal11,KrauEtal13}. These findings point towards a broader role of hippocampal function in the rodent beyond providing a pure map of space and into the realm of memory function.

The work I described in Chapter 2 builds off this line of thinking by investigating how the presupposed spatial representation system of the hippocampus interfaces with human episodic memory abilities. By showing that place cells are a neural mechanism allowing for memory episodes to become ``tagged'' with the location where they occurred, I revealed a concrete link between the spatial coding properties of the hippocampal formation and episodic memory function. This integration further strengthens the hypothesis that hippocampal activity can be conceptualized not just in terms of its relation to external space, but rather as part of a general engine of episodic memory, of which space is a core component.

\section{From rodents to humans}

At its core, a major aim of my work is to test theories of MTL function that are derived from rodent data with neural activity recorded directly from humans. As I've outlined, there are certainly many interspecies commonalities, such as the place cells I've shown in Chapter 2, the EC path equivalent cells I discussed in Chapter 3, and the presence of movement related oscillatory activity highlighted in Chapter 4. However, in spite of these similarities, key differences have emerged. In particular, hippocampal movement related oscillations appear at different frequencies in the rodent and in humans. In humans, this signal seems to exist between 2 and 4 Hz, much slower than the traditional theta signal in rodents. My findings are consistent with converging results indicating that human hippocampal theta is present at lower frequencies than in rodents \citep{WatrEtal13a,Jaco14}, and raises concerns about the applicability of rodent-derived models of brain function that rely the precise frequency of the theta oscillation \citep{JensLism98,BurgEtal07}. Whether the presence of a slower theta signal in humans truly has functional relevance remains to be seen. It could be the case that the decrease in frequency is simply a byproduct of our larger brains with an increased number of neurons \citep{BuzsDrag04}. Alternatively, because of the relationship of theta to the coordination of cell assemblies, a slower frequency signal has been theorized to allow the human brain to bind together larger sets of neurons to facilitate larger memory capacity \citep{Jaco14}.


% but also differences. Theta. Memory effects that can only be tested in humans.

% memory --> distinction between space and memory --> reconcilation Buzsaki mozer it's all the same just different



% and, moreover, rodent work has shown that chemically induced inactivation of the theta signal disrupts grid cell firing patterns \citep{KoenEtal11,BranEtal11}.


% At its core, my work has sought to provide a valuable link between our detailed understanding of MTL function with regards to spatial navigation and memory derived from rodent and 

% my work highlights some key distinctions between rodents and humns
% growing evidence that theta is slower

% 

\section{Future directions and concluding remarks}

Throughout this dissertation, the historical distinction between the spatial coding properties of the medial temporal lobe studied primarily in rodents and the memory-centric properties studied primarily in humans has framed much of my work. How do animals maintain a neural representation of their environment? How do humans encode and retrieve new memories? Given that these two phenomena make use of the same neural structures, are they actually distinct systems, or are we simply viewing the same system through two different lenses? This idea that the navigation ``system'' and the memory ``system'' are, in fact, part of a largely unitary construct is not new, but it is only recently that precise explanatory theories have been put forth. In one view, the neural representation of space in the rodent is an evolutionary precursor to declarative memory abilities in humans \citep{BuzsMose13}. Here, the ability to store and retrieve memories relies on the same neural computations responsible for keeping track of location in space. In another view, the hippocampus is thought of not in spatial terms or in memory terms, but rather as being the seat of relational processing \citep{CoheEich93,Eich14}. The discovery that many hippocampal cells keep track of time in addition to location \citep{MacDEtal11,PastEtal08} provides the neural circuitry necessary to bind co-occurring events together in both time and space.

Regardless of the precise mechanism, I believe my work serves to further this view that the neural representations of space and the neural mechanisms underlying memory function are not as distinct from one another as previously thought. The ability to study direct neural activity from humans allows to us to build upon the important work done in the rodent domain, and it also, importantly, allows us to test spatial navigation and memory function simultaneously in ways that are not possible in rodent research. There is, of course, still much that remains unanswered. In the final section, I briefly highlight some areas of potential future investigation.

\paragraph{Intra-hippocampal distinctions in spatial coding}
Research investigating the spatial coding properties of the human hippocampus, has, thus far, treated the hippocampus as a single structure, implicitly assuming that the hippocampus is uniform in its function. This is, unsurprisingly, not entirely true. Anatomically, the hippocampus can be broken down into a number of subfields, in particular CA1, CA3, and dentate gyrus \citep{AndeEtal06}. Rodent work has shown that, while all of these regions contain spatially selective cells, they are dissociable in their behaviors, specifically with regards to place cell remapping \citep{LeutEtal04,LeutEtal05a,LeutEtal07}. In addition to this architectural delineation, the hippocampus exhibits a functional gradient along its longitudinal axis, such that the rodent dorsal hippocampus is associated with more precise spatial turning and the ventral hippocampus is associated with a coarser representation of space \citep{KjelEtal08,StraEtal14}.

The reason for the oversimplification of hippocampal anatomy in humans is largely technical -- it has been difficult to tell exactly where in the hippocampus electrodes are placed. However, recent work combining advanced computational algorithms and higher resolution MRI techniques has made precise electrode localizations possible \citep{YushEtal15,DyksEtal12}. Going forward, mapping human hippocampus with the same level of anatomical detail as done in the rodent will be a valuable step in our understanding of hippocampal function.


% ca1 ca3 dg function, remapping
% along the axis. place field size. spatial coding precesion. connections iwth ec.

\paragraph{The role of theta phase in spatial coding, memory, and neuronal communication}
My work has focused on the amplitude of oscillatory activity, but the phase of the signal provides an additional source of information. Specifically, the phase of the theta oscillation has been shown to be tightly coupled to many aspects of neural computation and communication. In many ways, theta is thought to be a timing signal that coordinates that activity of assemblies of neurons \citep{Buzs05}. Place cell activity is tied to the phase of the ongoing theta rhythm in rodents \citep{OKeeRecc93,SkagEtal96}, and the phase of the theta oscillation contains complementary information about a rodent's location in the environment \citep{AgarEtal14}. Beyond spatial coding, the phase of an oscillation is related to synaptic plasticity \citep{BuzsDrag04,HuerLism93} and, through the phenomenon known as phase-amplitude coupling (where the phase of theta is coupled to the amplitude of a higher frequency oscillation), involved in inter-region communication \citep{ColgEtal09}.

In humans, the phase of oscillatory signals is beginning to receive more attention. The spiking activity of human hippocampal neurons is often ``phase-locked'' to low frequency oscillations \citep{JacoEtal07}, which is predictive of successful memory formation \citep{RutiEtal10}. Phase-amplitude coupling has also been shown to correlate with successful memory formation \citep{CanoEtal06,LegaEtal14}. The role of phase and its relationship to the neural representation of space, however, has not yet been directly studied in humans. Thus, aligning phase analyses across rodent and human data represents a relatively new and promising avenue of research.

Finally, 

% phase precession
% josh mentioned theta, inter region communication. gamma. colgin. phase/amp coupling 

% \paragraph{yes}
% vision / eye tracking













