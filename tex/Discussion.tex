% !TEX root = ../thesis.tex
\chapter{General discussion}
\large

\section{Summary}

In this dissertation, I recorded intracranial brain activity as patients traveled through carefully designed 3D virtual environments in order to investigate the neural signals related to spatial navigation and memory function. My work shows that the neural mechanisms responsible for representing external space are also involved in the storage and retrieval of episodic memories, helping to reconcile how the hippocampus and medial temporal lobe (MTL) can simultaneously be vital for both navigation and declarative memory abilities. I have also advanced our understanding of exactly how space is coded in the human brain using both single unit and oscillatory analyses, and I provide insights into how this neural system aligns with or differs from the more well characterized rodent brain. 

In Chapter 2, I isolated place cells in the human MTL and examined whether their functional role extends beyond simply supporting a representation of space. Place cells have been identified in humans in only a small number of previous studies \citep{EkstEtal03,JacoEtal10}, thus the identification of place cells was a crucial first step in my analysis. We found that $\sim$25\% of MTL neurons had well defined place fields, such that they exhibited significantly elevated rates of spiking within the field compared to other areas of the virtual environment. Based upon models of human episodic memory function, we hypothesized that these place cells would not only involved in coding for space, but should also reactivate when patients recalled events that occurred within the virtual environment. This hypothesis is grounded a theoretical account of human memory known as \textit{retrieved context}, which argues that when a memory is stored, it is stored along with its associated context. Here, context refers to an amalgam of both internal and external information present when an event occurs \citep{McGe42,Bowe72}, including, but not limited to, space and time (hence, the term spatiotemporal context). When a memory is later recalled, the theory posits that the context is recovered as well \citep{HowaKaha02a,PolyEtalTulv,LohnKaha13a}. If place cells represent not just an internal map, but moreover the \textit{spatial context} of a memory, then the pattern of place cell activity that represents a particular location should reemerge when events that occurred at that location are later recalled, even when the participant is no longer navigating the environment. Indeed, this is precisely what we find, such that place cell activity prior to and during the vocalization of a memory closely resembles the place cell activity that codes for the location where the memory was formed.



\section{The dual role of the hippocampus: memory and space}
% In the rodent literature, there as been a longstanding debate about the exact 

\section{Regional differences in spatial coding}

\section{From rodents to humans}
% josh mentioned theta, inter region communication

\section{Future directions and concluding remarks}