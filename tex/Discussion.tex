% !TEX root = ../thesis.tex
\chapter{General discussion}
\large

\section{Summary}

In this dissertation, I recorded intracranial brain activity as patients traveled through carefully designed 3D virtual environments in order to investigate the neural signals related to spatial navigation and memory function. My work shows that the neural mechanisms responsible for representing external space are also involved in the storage and retrieval of episodic memories, helping to reconcile how the hippocampus and medial temporal lobe (MTL) can simultaneously be vital for both navigation and declarative memory abilities. I have also advanced our understanding of exactly how space is coded in the human brain using both single unit and oscillatory analyses, and I provide insights into how this neural system aligns with or differs from the more well characterized rodent brain. 

In Chapter 2, I isolated place cells in the human MTL and examined whether their functional role extends beyond simply supporting a representation of space. Place cells have been identified in humans in only a small number of previous studies \citep{EkstEtal03,JacoEtal10}, thus the identification of place cells was a crucial first step in my analysis. We found that $\sim$25\% of MTL neurons had well defined place fields, such that they exhibited significantly elevated rates of spiking within the field compared to other areas of the virtual environment. Based upon models of human episodic memory function, we hypothesized that these place cells would not only be involved in coding for space, but should also reactivate when patients recalled events that occurred within the virtual environment. This hypothesis is grounded in a theoretical account of human memory known as \textit{retrieved context}, which argues that when a memory is stored, it is stored along with its associated context. Here, context refers to an amalgam of both internal and external information present when an event occurs \citep{McGe42,Bowe72}, including, but not limited to, space and time (hence, the term spatiotemporal context). When a memory is later recalled, the theory posits that the context is recovered as well \citep{HowaKaha02a,PolyEtalTulv,LohnKaha13a}. If place cells represent not just an internal map, but, moreover, the \textit{spatial context} of a memory, then the pattern of place cell activity that represents a particular location should reemerge when events that occurred at that location are later recalled, even when the participant is no longer navigating the environment. Indeed, this is precisely what we find, such that place cell activity prior to and during the vocalization of a memory closely resembles place cell activity that codes for the location where the memory was formed.

The work described in Chapter 2 capitalized on the knowledge that place cells exist in human MTL and exhibit similar behavior to place cells found in rodents. However, our understanding of the intricacies of the neural representation of space in the human brain still lags far behind that of rodents, where numerous types of spatially tuned cells have been found and the distinctive functions of MTL subregions have been fairly well characterized (for a review of the rodent literature, see \cite{MoseEtal08}). In Chapter 3, I attempted to partially close this gap by identifying additional classes of specialized cells beyond place cells in humans, as well as by highlighting key differences in the type of information coded by distinct MTL subregions. We first isolated cells in the MTL whose firing rates varied as a function of location within the virtual environment, and then we further sorted these cells based on the precise characteristics of their firing patterns. We found that cells predominantly located in the entorhinal cortex (EC) coded for multiple related locations in the virtual environment, in contrast to the standard place cell that  fires at a single location. More specifically, these EC cells exhibited multiple firing fields such that individual cells spiked at the same relative location across multiple distinct regions of the environment, seemingly coding a measure of relative distance that is tied to environment's geometry. We labeled these cells \textit{path equivalent} cells, as their properties are very similar to path equivalent cells previously found in rodent EC \citep{FranEtal00}. This finding nicely parallels the rodent literature, where there exists a diversity of cell classes in the EC that code for numerous environmental features \citep{HaftEtal05,FyhnEtal04,DerdEtal09,SargEtal06,SolsEtal08}.

In Chapter 4, I continued my comparison between rodent and human data, moving from analyses of neuronal spiking to analyses of oscillatory activity. Focusing solely on the hippocampus, I investigated whether the \textit{theta} oscillation, which plays so prominent a role in rodent spatial navigation, was similarly modulated by movement during human navigation. Though previous human studies have shown a general increase in low frequency hippocampal activity during periods of virtual movement \citep{CaplEtal03,EkstEtal05,JacoEtal10c}, my task was explicitly designed to test for the effects movement speed on neural activity, which is tightly coupled to theta power in the rodent \citep{Vand69}. I found that movement related oscillatory activity in the human hippocampus is consistently expressed at frequencies between 2 and 4 Hz, unlike the functionally analogous rodent signal which appears at $\sim$7--8 Hz \citep{Buzs02}. Moreover, I showed that the amplitude of the theta signal increases as a function of movement speed, replicating the longstanding rodent phenomenon \citep{McFaEtal75}. These results are consistent with converging findings indicating that human hippocampal theta is present at lower frequencies than in rodents \citep{WatrEtal13a,Jaco14}, and raises concerns about the applicability rodent-derived models of brain function that rely the precise frequency of the theta oscillation \citep{JensLism98,BurgEtal07}.

%In addition to movement related oscillatory activity, I also examined the relationship between low frequency activity and memory for object locations within the virtual environment. I found that a ``slow theta'' signal present at $\sim$3 Hz was correlated with better memory performance whereas a ``fast theta'' at $\sim$7 Hz was correlated with worse memory performance, revealing a 


\section{Beyond space in the hippocampus}
In the rodent literature, there has been a longstanding debate about the exact nature of the hippocampus. Namely, does the hippocampus act, fundamentally, as cognitive map that encodes the surrounding spatial features, or has the fact that many hippocampal cells are so strongly tuned to space caused researchers to overlook non-spatial coding properties of the hippocampus \citep{EichEtal99}? The cognitive mapping hypothesis, popularized by \citet{OKeeNade78} held great sway in the field for a number of decades, but more recent work has begun to reveal that cells in the rodent hippocampus are modulated by many variables beyond spatial information. Hippocampal cells have been shown to respond to non-spatial factors such as texture, odor, and color \citep{YounEtal94,WoodEtal99,LeutEtal05,IgarEtal14}. Moreover, the firing patterns of place cells are are affected by task demands, such that they distinguish between future goal locations \citep{WoodEtal00,FerbShap03}, conjunctively code for item and place information \citep{KomoEtal09}, and can keep track of temporal order \citep{MannEtal07}. These findings point towards a broader role of hippocampal function in the rodent beyond providing a pure map of space and into the realm of memory function.

The work I described in Chapter 2 advances this line of thinking -- investigating the spatial representation system of the hippocampus in non-spatial settings -- into the study of human memory. By showing that place cells are a neural mechanism allowing for memory episodes to become ``geotagged'' with the location where they occurred, I revealed a concrete link between the spatial coding properties of the hippocampal formation and episodic memory function. This integration further heightens the hypothesis that the hippocampus can be thought of not just in terms of its relation to external space, but rather as part of general engine of episodic memory, of which space is a core component.

% chapter 3?

\section{Regional differences in medial temporal lobe spatial coding}

What are the distinctive roles of various MTL subregions, in particular, the hippocampus and the entorhinal cortex, in coding for space? As mentioned in Chapter 2 and highlighted in Chapter 3, there are many different types of spatially tuned cells. Hippocampal place cells create a seemingly allocentric (map based) representation of external space \citep{OKeeDost71,Mull96}. This representation is formed based upon features of the environment, such as salient landmarks \citep{OKeeBurg96}. Notably, when a rodent is moved between environments or aspects of the current environment are changed, the hippocampal spatial code undergoes a process known as \textit{remapping} \citep{MullKubi87}. When remapping occurs, a place cell that was active in one environment may not be active in another, or it may alter the location of its place field to represent a completely new area of space \citep{MarkEtal95a,LeutEtal04a,LeutEtal05}. 

n contrast, cells in upstream brain regions, like the EC, tend to provide a quantitatively different sort of information, such that they do not code for precise locations but instead 

% Given that many EC neurons synapse directly onto hippocampal place cells \citep{ZhanEtal13}, it is believed the the location specificity of place cells in the hippocampus may be facilitated by the transmission of a diverse array of EC input \citep{MoseMose13}, a hypothesis which is compatible with my findings.

% not just spatial coding
% mtl differentiation
% spatial/item stuff
% broad -- specific
% i don't know

\section{From rodents to humans}
% my work highlights some key distinctions between rodents and humns
% growing evidence that theta is slower

% 

\section{Future directions and concluding remarks}

% hippocampal subfield/axis 
% josh mentioned theta, inter region communication
% vision / eye tracking











