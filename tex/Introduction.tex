% !TEX root = ../thesis.tex
\chapter{Introduction}
% \large


% What does the intro need to cover?
%
% Human memory. episodic. hippocampus.
% Rodent spatial navigation backround. Place cells. Theta.
%
%
% Stupid everyday example of memory in space. How do we do this? To understand, we need to bring together two fields that are often distinct.

As we move through our lives, we also move through the world. We go from room to room within our homes, we commute from our homes to our workplace, we travel throughout our neighborhoods and cities, and we take vacations to explore new countries. As we move, we learn our way around our environment. We create an internal map - an internal representation of external space - that allows us to get from place to place, and, in addition, provides us with the means to keep track of where the events that we experience in our lives occur. We have the ability to both navigate our world, and to form rich detailed memories as we move through it. Somehow, our brains support these dual abilities, which are fundamental for our daily quality of life.

Decades of neuroscience research have attempted to uncover the neural mechanisms responsible for spatial navigation and memory function, yet these two fields have largely been studied independently from each other. The investigation into the neural basis of spatial navigation was propelled forward over 40 years ago by the discovery of \textit{place cells} in ambulating rats \citep{OKeeDost71}. Place cells are neurons, primarily located in the hippocampus, that fire action potentials whenever the animal is physically located at a particular spot in world. They are remarkable in that they reveal a clear mapping of the external environment onto a neural substrate, and they provided the first indication as to how the brain might create an internal, or cognitive, map of our surroundings. This finding, which has since earned discoverer John O'Keefe a nobel prize (CITE), spawned an entire subfield of neuroscience focused on the detailing the neural representation of space in the rodent hippocampus and the surrounding medial temporal lobe structures of the brain.

In contrast, our understanding of the neural underpinnings of memory function is rooted in human neurophysiology. When patient H.M.\ had his hippocampi removed to treat his medication-resistant epilepsy, he could no longer reliably form new memories \citep{ScovMiln57}. More specifically, he could no longer form new \textit{episodic} memories, or memories for experienced events \citep{Tulv72}. As with rodents and  place cells, this discovery of a clear functional relationship between a neural structure -- once again, the hippocampus -- and an observable behavior served to focus much of the field's future attention. In this case, it was focused on the role of the human hippocampus in supporting episodic memory function.

In spite of this historical delineation, the space we inhabit and the memories we form within those spaces are intricately linked. When we reimagine the past, be it what we had for breakfast today or a favorite childhood birthday party from decades prior, the location of the past event is one of the primary and salient features of the memory. A main goal of the work described in this dissertation is to bridge the divide between these areas of research, spatial navigation in rodents and memory for past events in humans, and simultaneously study spatial navigation and memory function using human neural recordings. In order to do so, I will first describe certain methodological details upon which my work is based, including how the neural data is collected and how human navigation and memory can be probed in an experimental setting.

% Despite the
% As will be discussed, the reasons for this delineation are both historical and the result methodological limitations that have only recently been overcome. A primary goal of the work described in this dissertation is to bridge this divide and simultaneously study spatial navigation and memory function using human neural recordings.

% The neuroscience of memory 



% focus on hippocampus/mtl
% ieeg / microelectrodes
% episodic memory
% spatial navigation
% theta
% vr


% When we reimagine the past, be it what we had for breakfast today or a favorite childhood birthday party from decades prior, the location of the past event is one of the primary and salient features of the memory. 


\section{Human intracranial recordings}

Rodent place cells were discovered by recording \textit{in vivo} activity from individual neurons using electrodes inserted into rodent hippocampus. To facilitate making comparisons to rodent research, the studies performed here all make use of the rare opportunity to collect intracranial recordings of neural activity from electrodes implanted directly in the human brain. For clear ethical reasons, recording neural activity from the general population using direct brain implants is not justifiable. As a result, this work relies on data from the population of epilepsy patients who are undergoing monitoring for seizure localization. These patients all suffer from seizures that cannot be controlled though pharmacological intervention, and they have elected to have electrodes placed subdurally (on the surface of the cortex) as well as to have electrode depth probes inserted into deeper brain structures, such as the hippocampus.

The goal of this procedure is to localize the area of the brain responsible for the generation of seizures, with the aim of ultimately ressecting the seizure focus to reduce or eliminate seizure production. Patients remain in the hospital for $\sim$1--3 weeks while their brain activity is being recorded. The data collected using this method, known as either intracranial electroencephaologic (iEEG) or electrocorticographic (ECoG) recordings, provide the clinicians with a means of assessing which areas of the brain are epileptogenic, and the procedure results in significant reduction in seizure frequency in approximately 50\% of patients \citep{Kaha06}. 

When patients are not occupied with clinical matters, they are often free to participate in cognitive studies, affording researchers an otherwise \hl{unfeasible} opportunity to measure the electrical activity of the brain recorded via direct contact with active tissue. These data can be broken down into two distinct categories. First, one can analyze the signals recorded from macroelectrodes, which are relatively large contacts with diameters measured in the range of millimeters. These contacts record the activity of a $\sim$4mm$^{2}$ area of tissue, encompassing the summed activity of many \hl{number} of neurons. Signals recorded from these macroelectrodes are generally analyzed by either examining how voltage changes as a function of experiment condition, or by decomposing the voltage time series into its constituent frequencies and determining how the power (amplitude squared) and phase of the signal at various frequencies is related to experimental conditions of interest.

The second type of data is collected from microelectrodes, which are much smaller than macroelectrodes, having diameters in the range of microns (in the case of the studies performed here, the microelectrodes have a diameter of 40$\mu$m). Microelectrodes extend from the tip of the depth probes inserted into deep brain structures, and, importantly, are small enough to measure the action potentials of individual neurons. Here, the unit of analysis is not necessarily the voltage or power signal, but is rather the timing and rate of neuronal spiking. This spiking data is analogous to the type of data that led to the discovery of place cells in rodents, allowing for similar analytic methods to be carried out on human data. 

THIS IS UGH

\hl{MAKE FIGURE WITH VOLATE TRACE HERE}

% As a result, non-invasive techniques to study the relationship between brain activation and cognition such as functional magnetic resonance imaging (fMRI) dominate the field of cognitive neuroscience. However, while fMRI has proved to be an invaluable, and due to to its accessibility, very popular,  neuroscience tool, its utility is limited 


\section{The neural representation of space}
\section{Episodic memory and spatiotemporal context}
\section{From rodents to humans}
\section{Overview}

In chapter 2, I take what (limited) knowledge of what we know about the human neural representation of space and show how it fits into models of memory function.

In chapter 3, I extend our knowledge of the human neural representation of space, which further links rodent and human literature and gives us insight into the precise roles of MTL regions. New cell type is a key advancement of the field.

In chapter 4, I more precisely compare core findings from rodent navigation with human data fom a optimized task. some similarities, but some differences with important theoretical implications.

- Chapter 2: spatial components of episodic memories. context reinstatement.
- Chapter 3: ec repeating spatial activations. non-specific vs hipp specific.
- Chapter 4: role of theta in movement compared to rodents. Also memory..
