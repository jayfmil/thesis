% !TEX root = ../thesis.tex
\chapter{Introduction}



% What does the intro need to cover?
%
% Human memory. episodic. hippocampus.
% Rodent spatial navigation backround. Place cells. Theta.
%
%
% Stupid everyday example of memory in space. How do we do this? To understand, we need to bring together two fields that are often distinct.

As we move through our lives, we also move through the world. We go from room to room within our homes, we commute from our homes to our workplace, we travel throughout our neighborhoods and cities, and we take vacations to explore new countries. As we move, we learn our way around our environment. We create an internal map - an internal representation of external space - that allows us to get from place to place, and, in addition, provides us with the means to keep track of where the events that we experience in our lives occur. We have the ability to both navigate our world, and to form rich detailed memories as we move through it. Somehow, our brains support these dual abilities, which are fundamental for our daily quality of life.

Decades of neuroscience research have attempted to uncover the neural mechanisms responsible for spatial navigation and memory function, yet these two fields have largely been studied independently from each other. The investigation into the neural basis of spatial navigation was propelled forward over 40 years ago by the discovery of \textit{place cells} \citep{OKeeDost71} in ambulating rats. Place cells are neurons, primarily located in the hippocampus, that fire action potentials whenever the animal is physically located at a particular location in world. They are remarkable in that they reveal a clear mapping of the external environment onto a neural substrate, and they provided the first indication as to how the brain might create an internal, or cognitive, map of the world. This finding, which has since earned discoverer John O'Keefe a nobel prize (CITE), spawned an entire subfield of neuroscience focused on the detailing the neural representation of space in the rodent hippocampus and the surrounding medial temporal lobe.

In contrast, our understanding of the neural underpinnings of memory function is rooted in human neurophysiology. When patient H.M.\ had his hippocampi removed to treat his epilepsy, he could no longer reliably form new memories \cite{ScovMiln57}. More specifically, he could no longer form new \textit{episodic} memories, or memories for experienced events \cite{Tulv72}.

In contrast, our knowledge of the neural basis of spatial navigation is mainly derived from work with rodents, whereas our understanding of the neural underpinnings of memory function is rooted in human neurophysiology. As will be discussed, the reasons for this delineation are both historical and the result methodological limitations that have only recently been overcome. A primary goal of the work described in this dissertation is to bridge this divide and simultaneously study spatial navigation and memory function using human neural recordings.

The neuroscience of memory 



focus on hippocampus/mtl
ieeg / microelectrodes
episodic memory
spatial navigation
theta
vr


% When we reimagine the past, be it what we had for breakfast today or a favorite childhood birthday party from decades prior, the location of the past event is one of the primary and salient features of the memory. 


\section{Human intracranial recordings}



\section{The neural representation of space}
\section{Episodic memory and spatiotemporal context}
\section{From rodents to humans}
\section{Overview}

In chapter 2, I take what (limited) knowledge of what we know about the human neural representation of space and show how it fits into models of memory function.

In chapter 3, I extend our knowledge of the human neural representation of space, which further links rodent and human literature and gives us insight into the precise roles of MTL regions. New cell type is a key advancement of the field.

In chapter 4, I more precisely compare core findings from rodent navigation with human data fom a optimized task. some similarities, but some differences with important theoretical implications.

- Chapter 2: spatial components of episodic memories. context reinstatement.
- Chapter 3: ec repeating spatial activations. non-specific vs hipp specific.
- Chapter 4: role of theta in movement compared to rodents. Also memory..
