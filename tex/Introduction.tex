\chapter{Introduction}



What does the intro need to cover?

Human memory. episodic. hippocampus.
Rodent spatial navigation backround. Place cells. Theta.


Stupid everyday example of memory in space. How do we do this? To understand, we need to bring together two fields that are often distinct. 








- Chapter 2: spatial components of episodic memories. context reinstatement.
- Chapter 3: ec repeating spatial activations. non-specific vs hipp specific.
- Chapter 4: role of theta in movement compared to rodents. Also memory..


- What is the thesis really about?
- How the human brain supports spatial navigation
- Integration of episodic memory and spatial navigation (what happened where?)
- Is this about memory? Space? Physiology? What frames the whole thing?
- How does the train task fit into this?
- Can't be overly cognitive

- If it's what happened where, then it more is about memory. I'm starting to think it should be about ``space''. How space is coded, how it integrates with the memory system.

- 




\section{The neural representation of space}
\section{From rodents to humans}
\section{Overview}

In chapter 2, I take what (limited) knowledge of what we know about the human neural representation of space and show how it fits into models of memory function.

In chapter 3, I extend our knowledge of the human neural representation of space, which further links rodent and human literature and gives us insight into the precise roles of MTL regions. New cell type is a key advancement of the field.

In chapter 4, I more precisely compare core findings from rodent navigation with human data fom a optimized task. some similarities, but some differences with important theoretical implications.

- Chapter 2: spatial components of episodic memories. context reinstatement.
- Chapter 3: ec repeating spatial activations. non-specific vs hipp specific.
- Chapter 4: role of theta in movement compared to rodents. Also memory..
