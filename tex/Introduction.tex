% !TEX root = ../thesis.tex
\chapter{Introduction}
% \large


% What does the intro need to cover?
%
% Human memory. episodic. hippocampus.
% Rodent spatial navigation backround. Place cells. Theta.
%
%
% Stupid everyday example of memory in space. How do we do this? To understand, we need to bring together two fields that are often distinct.

As we move through our lives, we also move through the world. We go from room to room within our homes, we commute from our homes to our workplace, we travel throughout our neighborhoods and cities, and we take vacations to explore new countries. As we move, we learn our way around our environment. We create an internal map - an internal representation of external space - that allows us to get from place to place, and, in addition, provides us with the means to keep track of where the events that we experience in our lives occur. We have the ability to both navigate our world, and to form rich detailed memories as we move through it. Somehow, our brains support these dual abilities, which are fundamental to our daily quality of life.

Decades of neuroscience research have attempted to uncover the neural mechanisms responsible for spatial navigation and memory function, yet these two fields have largely been studied independently from each other. The investigation into the neural basis of spatial navigation was propelled forward over 40 years ago by the discovery of \textit{place cells} in ambulating rats \citep{OKeeDost71}. Place cells are neurons, primarily located in the hippocampus, that fire action potentials whenever the animal is physically located at a particular spot in world. They are remarkable in that they reveal a clear mapping of external space onto a neural substrate, and they provided the first indication as to how the brain might create an internal, or cognitive, map of our surroundings. This finding, which has since earned discoverer John O'Keefe the 2014 Nobel prize in physiology or medicine, spawned an entire subfield of neuroscience focused on detailing the neural representation of space in the rodent hippocampus and the surrounding medial temporal lobe structures of the brain.

In contrast, our understanding of the neural underpinnings of memory function is rooted in human neurophysiology. When patient H.M.\ had his hippocampi removed to treat his medication-resistant epilepsy, he could no longer reliably form new memories \citep{ScovMiln57}. More specifically, he could no longer form new \textit{episodic} memories, or memories for experienced events \citep{Tulv72}. As with rodents and  place cells, this discovery of a clear functional relationship between a neural structure -- once again, the hippocampus -- and an observable behavior served to focus much of the field's future attention. In this case, it was focused on the role of the human hippocampus in supporting episodic memory function.

In spite of this historical delineation, the space we inhabit and the memories we form within those spaces are intricately linked. When we reimagine the past, be it what we had for breakfast today or a favorite childhood birthday party from decades prior, the location of the past event is one of the primary and salient features of the memory. A main goal of the work described in this dissertation is to bridge the divide between these areas of research, spatial navigation in rodents and memory for past events in humans, and simultaneously study spatial navigation and memory function using human neural recordings. In order to do so, I will first describe certain methodological details upon which my work is based, including how the neural data is collected and how human navigation and memory can be probed in an experimental setting.

% Despite the
% As will be discussed, the reasons for this delineation are both historical and the result methodological limitations that have only recently been overcome. A primary goal of the work described in this dissertation is to bridge this divide and simultaneously study spatial navigation and memory function using human neural recordings.

% The neuroscience of memory 



% focus on hippocampus/mtl
% ieeg / microelectrodes
% episodic memory
% spatial navigation
% theta
% vr


% When we reimagine the past, be it what we had for breakfast today or a favorite childhood birthday party from decades prior, the location of the past event is one of the primary and salient features of the memory. 


\section{Human intracranial recordings}

Rodent place cells were discovered by recording \textit{in vivo} activity from individual neurons using electrodes inserted into rodent hippocampus. To facilitate making comparisons to rodent research, the studies performed here all make use of the rare opportunity to collect intracranial recordings of neural activity from electrodes implanted directly in the human brain. For clear ethical reasons, recording neural activity from the general population using direct brain implants is not justifiable. As a result, this work relies on data from the population of epilepsy patients who are undergoing monitoring for seizure localization. These patients all suffer from seizures that cannot be controlled though pharmacological intervention, and they have elected to have electrodes placed subdurally (on the surface of the cortex) as well as to have electrode depth probes inserted into deeper brain structures, such as the hippocampus.

The goal of this procedure is to localize the area of the brain responsible for the generation of seizures, with the aim of ultimately ressecting the seizure focus to reduce or eliminate seizure production. Patients remain in the hospital for $\sim$1--3 weeks while their brain activity is being recorded. The data collected using this method, known as either intracranial electroencephaologic (iEEG) or electrocorticographic (ECoG) recordings, provide the clinicians with a means of assessing which areas of the brain are epileptogenic, and the procedure results in significant reduction in seizure frequency in approximately 50\% of patients \citep{Kaha06}. 

When patients are not occupied with clinical matters, they are often free to participate in cognitive studies, affording researchers a unique opportunity to measure the electrical activity of the brain recorded via direct contact with active tissue. The data collected with this methodology can be broken down into two distinct categories. First, one can analyze the signals recorded from macroelectrodes, which are relatively large contacts with diameters measured in the range of millimeters. These contacts record the activity of a $\sim$4mm$^{2}$ area of tissue, encompassing the summed activity of many large populations of neurons. Signals recorded from these macroelectrodes are generally analyzed by either examining how the voltage potential changes as a function of experiment condition, or by decomposing the voltage time series into its constituent frequencies and determining how the power and phase of the signal at various frequencies is related to experimental conditions of interest.

The second type of data is collected from microelectrodes, which are much smaller than macroelectrodes, having diameters in the range of microns (in the case of the studies performed here, the microelectrodes have a diameter of 40$\mu$m). Microelectrodes extend from the tip of the depth probes inserted into deep brain structures, and, importantly, are small enough to measure the action potentials of individual neurons. Here, the unit of analysis is not necessarily the voltage or power signal, but is rather the timing and rate of neuronal spiking. This spiking data is analogous to the type of data that led to the discovery of place cells in rodents, allowing for similar analytic methods to be carried out on human data. 

\hl{MAKE FIGURE WITH VOLATE TRACE HERE? or not necessary?}

% As a result, non-invasive techniques to study the relationship between brain activation and cognition such as functional magnetic resonance imaging (fMRI) dominate the field of cognitive neuroscience. However, while fMRI has proved to be an invaluable, and due to to its accessibility, very popular,  neuroscience tool, its utility is limited 

\section{The neural representation of space}

% The work described in this dissertations, in particular in Chapters 2 and 3,

When place cells were discovered and their properties first studied in the 1970s, researchers had their first glimpse into the precise neural mechanisms supporting an organism's ability to self-localize within the external environment. Place cells, with their well-defined ``place fields'' that link a physical location to an internal representation \citep{OKeeNade78}, suggested a  concrete mechanism by which the brain stores a map of the world. With a large enough set of place cells, each one coding for a particular region of space, an animal's location at any given moment can be represented by the activity of the population of cells. Indeed, if enough simultaneous place cells are recorded, a rodent's position can be accurately reconstructed from the neural activity \citep{WilsMcNa93,ZhanEtal98}. For a time, it seemed as though a pure cognitive map, as hypothesized by Edward Tolman decades prior \citep{Tolm48}, had been found.

Over time, however, it has become clear the spatial representation system in the brain is much more complex. Not only are there place cells in the hippocampus that code for particular locations, but there are numerous other types of cells throughout the medial temporal lobe (MTL) that are tuned to specific environmental properties. For example, head direction cells in pre- and parasubiculum code for the direction the animal is facing \citep{TaubEtal90}, border or boundary cells in the entorhinal cortex (EC) code for locations near the edge of an environment \citep{SolsEtal08}, and grid cells in the EC have non-localized firing fields that tessellate across an entire region of space \citep{HaftEtal05}.

In addition to these specialized cell types, brain oscillations -- the result of the summed activity of populations of neurons \citep{LachEtal03} -- are closely tied to rodent spatial navigation. Of particular interest is the hippocampal \textit{theta} oscillation, typically defined as 4--10 Hz, whose amplitude increases during periods of voluntary movement \citep{Vand69}. Moreover, theta is mechanistically related to the activity of spatially tuned cells. The theta oscillation is thought to play a key role in coordinating the timing of action potentials of place cells, such that the likelihood of firing varies as function of the phase of theta \citep{OKeeRecc93,SkagEtal96}. Oscillatory interference models of EC grid cell generation causally rely on theta when explaining how grid cells are formed \citep{BurgEtal07}. The diversity of spatially tuned cells and their modulation by oscillatory activity raises questions about how all of this information is integrated and what roles different MTL structures may play in supporting navigation abilities.

In contrast to the large volume of work describing the neural representation of space in the rodent brain, there has been relatively little work investigating the same questions in humans. Place cells and grid cells have indeed been shown to exist in human MTL \citep{EkstEtal03,JacoEtal10,JacoEtal13}, and low frequency oscillations are more prominent during periods of movement \citep{CaplEtal03,KahaEtal99}. However, our knowledge of how the human brain represents external space still pales in comparison to our knowledge of rodents. Do the properties of spatially tuned cells vary based on brain region? Does the theta oscillation, with its strong functional ties to navigation in the rodent, act in an analogous manner in humans? Furthermore, with so much of the MTL seemingly dedicated to spatial processing, reconciling how the same part of the brain is vital for memory function in humans has been a challenge. An aim of my work is to investigate these issues with the same level of detail as they have been investigated in rodents.


% with so much spatial specific stuff, what about memory?

% also theta

% also some human stuff, we will build off it

\section{Episodic memory and spatiotemporal context}

Space is an intuitively vital aspect of our episodic memories. The events that we experience in our lives necessarily occur at a particular time, with particular surroundings, and at a particular location. This can be framed more formally by saying that events take place within a \textit{spatiotemporal} context, such that we can remember both where and when an event occurred \citep{Tulv83,Eich04}. Yet, throughout the history of episodic memory experiments, the study of the role played by spatial information has largely been ignored. In general, episodic memory experiments present study participants with lists of words or pictures to encode so that they can later be recalled or recognized, devoid of any spatial information. Thus, a major drawback of these traditional methods is that they lack a sense of real world ecological validity, as it has been shown that spatial information is a key element in memory formation, organization, and retrieval \citep{MillEtal12a}. To overcome this limitation of previous studies, all of the work I will present makes use of virtual environments to extend experiments into three dimensions, allowing for the study of the neural representation of space, memory, and their interactions.

\paragraph{Virtual navigation} 

To study the neural representations of space in the rodent, researchers generally record from a freely moving animal that is tethered to a recording system. Unfortunately for human research, the ability to record from the human brain comes with the cost of limited mobility. When hospital patients perform iEEG experiments, they are either resting in their bed or sitting in their chair. To overcome this restriction, we create visually rich 3D environments that participants navigate on a laptop computer. While virtual tasks lack the locomotive and proprioceptive feedback of real navigation, previous studies of both rodent \citep{HarvEtal09,ChenEtal13} and human \citep{EkstEtal03,JacoEtal10,JacoEtal13} spatial navigation have revealed that virtual navigation is sufficient to elicit neural responses similar to actual navigation in rodents. In addition, though we lose certain attributes of real world navigation, we gain the capability to precisely design and control the virtual environments to our exact specifications.

The virtual environments that I will describe were created using the Panda Experiment Programming Library (PandaEPL), which is an open source 3D experiment platform \citep{SolwEtal13}. PandaEPL allows for the construction of freely navigable 3D environments, and importantly, incorporates extremely precise logging to accurately align patient activity in the task with the neural data.

% Much of the work discussed in this dissertation attempts to help us understand how the human brain supports episodic memory, with a focus on the spatial component of memory episodes.

% place cells can thought of in a memory orientated way

% \section{From rodents to humans}

% or is that here?

\section{Overview}

% episodic memory important, can only really be tested in humans
We are all defined by our memories and past experiences, of which space is a core component. This is perhaps most evident when considering the profound effects of neurodegenerative diseases such as Alzheimer's, where the loss of memory function, along with spatial disorientation, are primary symptoms. As we move through space and time, we are constantly encoding new memories for later retrieval, transforming the events of our lives into patterns of brain activity. Understanding the neural underpinnings of this process is a lofty yet fundamental goal of neuroscience.
% and it is a goal that can only truly be accomplished through the study of the human brain. Whether or not any animals possess episodic memory in a manner that is on par with humans is an open debate \citep{ClayEtal03}, yet is is hard to dispute the fact that only humans possess the combination of episodic memory, language, and communication abilities necessary for performing in-depth experiments of memory function.
Whether or not any animals possess episodic memory in a manner that is on par with humans is an open debate \citep{ClayEtal03}, yet is is hard to dispute the claim that this goal, due to our combination of memory, language, and communication abilities, can only truly be accomplished through the study of the human brain.

The body of work that I will present is focused on two overarching themes. First, my work serves to extend our knowledge of how the brain represents spatial information by studying spatial navigation in humans. Directly comparing rodent findings with data from human studies is necessary in order to validate theories of brain function that are derived mainly from the rodent literature. Second, I make use of the fact that memory in humans can be probed more explicitly than memory in animals to investigate how the brain represents the spatial components of memories, and what role these components may play in the general episodic memory engine.

In \textbf{Chapter 2}, I build upon our knowledge of the existence of place cells in humans to examine if place cells are simply an internal biomarker of location or whether they become integrated into a broader memory trace. To do so, I recorded the activity of individual medial temporal lobe neurons in patients performing a hybrid memory and spatial navigation task, and I found that place cells that were active during spatial navigation reactivated outside of navigation when patients recalled specific memory episodes. In \textbf{Chapter 3}, I add to our understanding of how the brain represents space through the discovery of new type of spatially tuned cell not previously seen humans. I show that \textit{path equivalent cells} encode distance traveled relative to environmental geometry, and, in line with rodent findings \citep{FranEtal00}, that these cells tend to be localized to the entorhinal cortex. In \textbf{Chapter 4}, I use analyses of oscillatory activity to investigate whether the key role played by theta oscillations in rodent navigation and memory is conserved across species. While I found that many core attributes translated from rodents to humans, there do exist fundamental inter-species differences in the behavior of movement related oscillatory activity, highlighting the need to validate existing theories using human data. In \textbf{Chapter 5}, I synthesize these findings and relate them to the broader literature of both spatial navigation and memory function.

% In chapter 3, I extend our knowledge of the human neural representation of space, which further links rodent and human literature and gives us insight into the precise roles of MTL regions. New cell type is a key advancement of the field.
% 
% In chapter 4, I more precisely compare core findings from rodent navigation with human data from a optimized task. some similarities, but some differences with important theoretical implications.

% - Chapter 2: spatial components of episodic memories. context reinstatement.
% - Chapter 3: ec repeating spatial activations. non-specific vs hipp specific.
% - Chapter 4: role of theta in movement compared to rodents. Also memory..
